\setupcolors     [state=start]
\setupinteraction[state=start,style=normal]

%% Layout : <<<
\setuplayout[
                   width=middle,
                  height=middle,
                %location=middle,
                topspace=0.5in,
             bottomspace=.75in,
          bottomdistance=.25in,
                  bottom=.25in,
               backspace=1.0in,
                cutspace=1.0in,
              leftmargin=0.55in,
             rightmargin=0.55in,
      leftmargindistance=0.1in,
     rightmargindistance=0.1in,
                  header=0.25in,
                  footer=0.5in,
           headerdistace=0.25in,
          footerdistance=0.25in,
                 marking=on,
%                   grid=yes,
    ]

\setuppagenumbering [location=footer]

%% >>>
%% Typescripts : <<<

\setupbodyfontenvironment[default][em=italic]

\setupbodyfont[dejavu,10pt]

%% >>>
%% Logos: <<<
\logo [TEX]      {Tex}
\logo [LATEX]    {Latex}
\logo [CONTEXT]  {Context}
\logo [PDFTEX]   {pdftex}
\logo [LUATEX]   {Luatex}
\logo [XETEX]    {Xetex}
\logo [MKII]     {MkII}
\logo [MKIV]     {MkIV}

\setupsorting[logo][style=normal]

%% >>>

\definetype[typeTEX][option=tex, style=type]
\definetype[command][color=darkred, style=type]
\definetype[options][color=darkblue, style=type]
\definetyping[TEX][option=tex, before=\startEXAMPLE,after=\stopEXAMPLE]
\definetyping[SIMPLETEX][option=tex]

\setupindenting[medium,yes]
\setupwhitespace[medium]

\setuphead[title][alternative=middle, textstyle=\ss\bf] 
\setuphead[section,subsubject,subsection]
          [numberstyle=\ss\bf,textstyle=\ss\bf]

\setuplistalternative[a]
    [distance=0pt,width=1em,stretch=10em,
     command=\hskip0.5em\ldots\hskip0.5em\relax]

\setuplist  [section]
            [margin=10em, alternative=a]

\useURL[practex][http://www.tug.org/pracjourn/2006-2/schmitz/]

\setupitemize[1][autointro]
\setupitemize[indenting=no]


%% Frames and Backgrounds : <<<
\definetextbackground
    [EXAMPLE]
    [           mp=background:random,
          location=paragraph,
     rulethickness=1pt,
        framecolor=darkred,
             width=broad,
            height=fit,
        leftoffset=1em,
       rightoffset=1em,
            before={\testpage[3]\blank[2*big]},
             after={\blank},
    ]

\startuseMPgraphic{background:random}
   path p;
   for i = 1 upto nofmultipars :
    p = (multipars[i]
     topenlarged 8pt
     bottomenlarged 4pt) randomized 4pt ;
   fill p withcolor lightgray ;
   draw p withcolor \MPvar{linecolor}
    withpen pencircle scaled \MPvar{linewidth};
   endfor;
\stopuseMPgraphic

\defineframedtext
    [EXAMPLEframe]
    [rulethickness=1pt,
        framecolor=darkred,
            height=6.55cm,
             width=broad,
        background=color,
   backgroundcolor=gray,
    ]

\defineoverlay[randomframe]
              [\useMPgraphic{background:random:frame}]

\startuseMPgraphic{background:random:frame}
   path p;
    p = (OverlayBox 
     topenlarged 10pt
     bottomenlarged 10pt) randomized 4pt ;
   fill p withcolor lightgray ;
   draw p withcolor \MPvar{linecolor}
    withpen pencircle scaled \MPvar{linewidth};
   endfor;
\stopuseMPgraphic
\setupexternalfigures[location={local,global,default}] 

%% >>>
%% Interface <<<

\definecolor [colorprettyone]   [r=.6,g=.0,b=.0]  % red
\definecolor [colorprettytwo]   [r=.0,g=.6,b=.0]  % green
\definecolor [colorprettythree] [r=.0,g=.0,b=.6]  % blue
\definecolor[colorprettyfour][orange]

\usemodule[int-load]
\loadsetups[cont-en.xml]
\loadsetups[t-simpleslides.xml]
\definetextbackground
    [setuptext]
    [           mp=background:random,
          location=paragraph,
     rulethickness=1pt,
        framecolor=darkgreen,
             width=broad,
        leftoffset=1em,
       rightoffset=1em,
             align=right,
            before={\testpage[3]\blank[2*big]},
             after={\blank\testpage[2]},
    ]

%% There gotta be a better way to configure this!

\unprotected\def\showSETUPrecord
  {\getvalue{\e!start setuptext}
     \tttf
     \nohyphens
     \veryraggedright
     \startXMLmapping [one]
       \doglobal\newcounter\currentSETUPargument
       \global\let\maximumSETUPargument\currentSETUPargument
       \bgroup
         \doif{\XMLpar{cd:command}{generated}{}}{yes}{\ttsl}%
         \doifelseXMLop{type}{environment}
           {\tex{\e!start}}{\startcolor[colorprettytwo]\tex{}}\ignorespaces
           \XMLflush{cd:sequence}\stopcolor\ignorespaces
       \egroup
       \doifelseXMLempty{cd:arguments}
         {}
         {\bgroup
            \setbox0=\hbox{\XMLflush{cd:arguments}}%
            \global\let\maximumSETUPargument\currentSETUPargument
            \doglobal\newcounter\currentSETUPargument
            \ignorespaces\XMLflush{cd:arguments}%
            \doif{\XMLpar{cd:command}{type}{}}{environment}
              {\hskip.5em\unknown\hskip.5em
               \doif{\XMLpar{cd:command}{generated}{}}{yes}{\ttsl}%
               \tex{\e!stop}\ignorespaces\XMLflush{cd:sequence}}%
            \endgraf
          \egroup
         %\bgroup
         %  \tx
         %  \doif{\XMLpar{cd:command}{interactive}{}}{yes}      {\quad INTERACTIVE}%
         %  \doif{\XMLpar{cd:command}{interactive}{}}{exclusive}{\quad INTERACTIVE ONLY}%
         %\egroup
        \startXMLmapping [two]
          \bgroup
            \doglobal\newcounter\currentSETUPargument
            \blank[\v!line]
            %\switchtobodyfont[small] % kan sneller
            \ignorespaces\XMLflush{cd:arguments}\endgraf
            %\endgraf
          \egroup
        \stopXMLmapping}%
     \stopXMLmapping
   \getvalue{\e!stop setuptext}}

\def\showSETUPnumber
  {\doglobal\increment\currentSETUPargument
   \hbox to 2em
     {\startcolor[blue]
      \ifcase\maximumSETUPargument\relax
        \or*\else\currentSETUPargument
      \fi\stopcolor
      \hss}}

\def\showSETUPassignment {\showSETUP
  {{\colorprettythree[}.\lower.5ex\hbox{=}.{\colorprettythree]}}
  {{\colorprettythree[}..,.\lower.5ex\hbox{=}.,..{\colorprettythree]}}}

\def\showSETUPkeyword {\showSETUP
  {\colorprettythree{[}...{\colorprettythree]}}
  {\colorprettythree{[}...,...{\colorprettythree]}}}

\def\showSETUPargument {\showSETUP
  {{\colorprettyone\leftargument}..{\colorprettyone\rightargument}}
  {{\colorprettyone\leftargument}..,...,..{\colorprettyone\rightargument}}}

\def\showSETUPdisplaymath {\showSETUP
  {\$\$...\$\$}
  {\$\$...\$\$}}

\def\showSETUPindex {\showSETUP
  {{\colorprettyone\leftargument}...{\colorprettyone\rightargument}}
  {{\colorprettyone\leftargument}..+...+..{\colorprettyone\rightargument}}}

\def\showSETUPmath {\showSETUP
  {\$...\$}
  {\$...\$}}

\def\showSETUPnothing {\showSETUP
  {...}
  {}}

\def\showSETUPfile {\showSETUP
  {~...~}
  {}}

\def\showSETUPposition {\showSETUP
  {(...)}
  {(...,...)}}

\def\showSETUPreference {\showSETUP
  {[...]}
  {[...,...]}}

\def\showSETUPcsname {\showSETUP
  {{\c!setup!command!{}}}
  {}}

\def\showSETUPdestination {\showSETUP
  {[{\colorprettyone\leftargument}..[ref]{\colorprettyone\rightargument}]}
  {[..,{\colorprettyone\leftargument}..[ref,..]{\colorprettyone\rightargument},..]}}

\def\showSETUPtriplet {\showSETUP
  {[x:y:z=]}
  {[x:y:z=,..]}}

\def\showSETUPword {\showSETUP
  {{\colorprettyone\leftargument}...{\colorprettyone\rightargument}}
  {{\colorprettyone\leftargument}.. ... ..{\colorprettyone\rightargument}}}

\def\showSETUPcontent {\showSETUP
  {{\colorprettyone\leftargument}...{\colorprettyone\rightargument}}
  {{\colorprettyone\leftargument}.. ... ..{\colorprettyone\rightargument}}}

%% >>>

\def\ShowStyle#1%
  {\blank[big]
   \midaligned{\startcombination[2*2]
     {\externalfigure[styles/#1][page=1,width=0.55\textwidth]}
     {Title Page}
     {\externalfigure[styles/#1][page=2,width=0.55\textwidth]}
     {Normal Slide}
     {\externalfigure[styles/#1][page=3,width=0.55\textwidth]}
     {Horizontal Picture}
     {\externalfigure[styles/#1][page=10,width=0.55\textwidth]}
     {Vertical Picture}
  \stopcombination}}

\starttext

\title{Simple Slides \\
  A \CONTEXT\ presentation module}

\startEXAMPLE
\placelist[section]
\stopEXAMPLE

\section{Introduction}

This module provides an easy|-|to|-|use interface
for creating simple slides/presentations in \CONTEXT. 
The salient features of this module are:
\startitemize
  \item The module is meant for presentations which will be shown on a
    digital projector. They have no interactive elements (such as buttons or
    hyperlinks) and no navigational tools (such as table of contents).
  \item The module comes with several predefined styles; these styles are sober
    in appearance and meant for academic presentations. It also provides some
    macros to help in presenting slides with both pictures and text.
  \item Most styles allow for some degree of user|-|reconfigurability. Designing
    a new style is also easy.
\stopitemize

This module provides a simple structure that will be suitable for beginning
or intermediate users of \CONTEXT, or someone who does not want to spend
too much time playing around with different configuration options for
\CONTEXT. As such it focusses on different users than Hans's presentation
modules that provide more and fancier features. This module also offers
much less features than the \LATEX\ \filename{beamer} package. Its main
strength is its ease of use; you should be able to write your first
presentation after spending five minutes with this manual.

\section{A bit of history}

The idea of a module suitable for simple presentations took shape when Thomas
started using \CONTEXT\ for preparing his course presentations. \CONTEXT\ comes
with a bunch of modules for presentations (the files \filename{s-pre-??.tex} in
\filename{$TEXMF/tex/context/base}) which are written by Hans Hagen. Hans
usually creates a new presentation style whenever he gives a talk about
\CONTEXT. As such, his presentation styles highlight the fancy and bleeding edge
features of \CONTEXT, and are not the most suitable starting point for academic
presentations. 

\CONTEXT\ does make creating your own presentation style relatively easy. So
Thomas wrote some presentation related macros (see the Prac\TEX\ article
{\tt\from[practex]}). With time, he extended these macros into a collection of
styles providing different visual effects, and later collected all of them in
the \filename{taspresent} module. He gave a talk about the
\filename{taspresent} module at the second \CONTEXT\ user meeting at Bohinj,
and in the ensuing discussions, Aditya and Thomas decided to modularize and
\quotation{\CONTEXT{ize}} some of the internals of the module, giving rise to
the current module. Most of the code in the current release has been
contributed by Aditya.

\section{Installation}

The module is installed in the usual way: simply unzip the archive
\filename{t-simpleslides-<date>.zip} into one of your \filename{$TEXMF}
trees, and from a terminal run \command{mktexlsr} (for \MKII) and
\command{mtxrun --generate} (for \MKIV). 

To verify that everything was installed correctly, run \command{kpsewhich t-simpleslides.tex} from a terminal (for \MKII) and \command{mtxrun --locate t-simpleslides.tex} (for \MKIV); these commands should return
the complete path of the files that you just installed.

\subsubject{A note about \TEX|-|engines}

We have extensively tested this module with \PDFTEX\ and \LUATEX\ (that is,
with \MKII\ and \MKIV). In spite of our best efforts, we have not been able to
get this module to work reliably with \XETEX. If you are a \XETEX\ guru, and
know how to fix some of the errors with \XETEX, we will appreciate the help.

\section{Quick start}

First you must tell \CONTEXT\ that you want to use this module. To do this
simply write:

\startTEX
\usemodule[simpleslides]
\stopTEX

The module sets the paper size and font sizes to values that are suitable for
presentations. Everything else is left like a default \CONTEXT\ document. The
module comes with the following styles that change the visual appearance of the
presentation.
\startitemize[columns, three]
  \item \options{BigNumber}
  \item \options{BottomSquares}
  \item \options{Boxed}
  \item \options{Ellipse}
  \item \options{Embossed}
  \item \options{Framed}
  \item \options{FramedTitle}
  \item \options{HorizontalStripes}
  \item \options{NarrowStripes}
  \item \options{PlainCounter}
  \item \options{RainbowStripe}
  \item \options{Rounded}
  \item \options{Shaded}
  \item \options{SideSquares}
  \item \options{SideToc}
  \item \options{Split}
  \item \options{Sunrise}
  \item \options{Swoosh}
  \item \options{ThickStripes}
\stopitemize

To use a style, say \options{BigNumber}, pass the \options{style=BigNumber}
option to the \filename{simpleslides} module:

\startTEX
\usemodule[simpleslides]
          [style=BigNumber]
\stopTEX

Some of the styles have a few variants that can be chosen using
\options{color} and \options{alternative} keys. These are explained in \in
{Section}[sec:styles].

By default, the Latin Modern Sans font is used. The module makes it easy to
use other fonts that come with a typical \TEX\ distribution. 
The following fonts are provided:
\startitemize[columns, three]
  \item \options{LatinModern}
  \item \options{LatinModernSans}
  \item \options{Bookman}
  \item \options{Chancery}
  \item \options{Gothic}
  \item \options{Helvetica}
  \item \options{Palatino}
  \item \options{Schoolbook}
  \item \options{Times}
\stopitemize

To choose a font, say \options{Helvetica}, pass \options{font=Helvetica} option
to the \filename{simpleslides} module as follows.
\startTEX
\usemodule[simpleslides]
          [style=BigNumber,
            font=Helvetica]
\stopTEX

The default font size is 17pt. Font size can be changed using 
the \options{size} key.

More details about the fonts, including information on how to use your own fonts
is given in \in{Section}[sec:fonts]. 

The complete setup for using this module is
\setup{simpleslides}

\subsubject{Structure of a presentation}

The \filename{simpleslides} module has a very simple model of a presentation. A
presentation consists of a title followed by a series of slides; the module
provides macros to help create a presentation title page and slides.
A minimal presentation is shown below. The result is shown in \in
{Figure}[fig:example].

\startEXAMPLE
\typefile[option=tex]{example.tex}
\stopEXAMPLE

\placefigure
  [top,bottom]
  [fig:example]
  {A minimal presentation}
  \startcombination[2*2]
    \startEXAMPLEframe[width=0.55\textwidth]
\startSIMPLETEX
  \usemodule[simpleslides]
            [style=BigNumber]
  \setupTitle[...]

  \starttext
  \SlideTitle{...}
  ...
  \SlideTitle{...}
  ...
  \stoptext
\stopSIMPLETEX
    \stopEXAMPLEframe
    {A minimal example}
    {\externalfigure[example][page=1,width=0.55\textwidth]}{Title page}
    {\externalfigure[example][page=2,width=0.55\textwidth]}{First slide}
    {\externalfigure[example][page=3,width=0.55\textwidth]}{Second slide}
  \stopcombination



\subsubject{Presentation title page}

A presentation title page displays the title of the presentation, the names of
the authors, and the date.
%% TODO: Also add institution and detail.
These can be specified using \typeTEX{\setupTitle} as follows:
\startTEX
\setupTitle
  [ title={Title of the presentation},
   author={Name of authors},
     date={Date of presentation},
  ]
\stopTEX

The macro \typeTEX{\placeTitle} places the title page in the presentation. It is
possible to change the look of \typeTEX{\placeTitle} using some additional
arguments to \typeTEX{\setupTitle}. These are explained in \in
{Section}[sec:setuptitle].

\subsubject{Presentation slide}

The \filename{simpleslides} module provides a \typeTEX{\SlideTitle} macro, which
starts a new slide (basically a new page), and typesets its argument as the
title of the slide. It also takes care of increasing the page counters and
progress bars, and setting up the background. The content of the slides follows
after this command.

A slide is a normal \CONTEXT\ page, so you can use any command or environment
that you want. Each presentation style sets up a style for itemizations, and
provides useful macros for placing pictures. These macros will be explained
later.



\section{Placing pictures}

If you want to place pictures in your slides, you can always use \CONTEXT's
\typeTEX{\externalfigure} macro. This module also provides a macro,
\typeTEX{\IncludePicture}, for preconfigured picture layouts. Two layouts are
provided:
\startitemize
  \item \options{horizontal}: the picture is placed under the title of the slide,
    so that it fits in the available space.
  \item \options{vertical}: the slide is divided into two columns; the picture is
    placed on the left column and text is placed on the right column.
\stopitemize
These layouts are shown in \in{Figure}[fig:pictures].

\placefigure
  [top,bottom]
  [fig:pictures]
  {Example of \options{horizontal} and \options{vertical} options for 
  \typeTEX{IncludePicture} macro}
  %FIXME: Change caption to \IncludePicture ..
  \startcombination[2*2]
    \startEXAMPLEframe[width=0.55\textwidth]
\startSIMPLETEX
  \usemodule[simpleslides]
            [...]
  \starttext
  ...
  \IncludePicture
      [horizontal]
      [cow]
      {A Dutch Cow}
  ...
  \stoptext
\stopSIMPLETEX
    \stopEXAMPLEframe
    {A horizontal picture}
    {\externalfigure[styles/BigNumber-blue]
                    [page=3,width=0.55\textwidth]}{A horizontal picture}
    \startEXAMPLEframe[width=0.55\textwidth]
\startSIMPLETEX
  \usemodule[simpleslides]
            [...]
  \starttext
  ...
  \IncludePicture
      [vertical]
      [mill]
      {The windmills are an example of a green energy source} 
  ...
  \stoptext
\stopSIMPLETEX
    \stopEXAMPLEframe
    {A vertical picture}
    {\externalfigure[styles/BigNumber-blue]
                    [page=10,width=0.55\textwidth]}{A vertical picture}
\stopcombination

A horizontal picture is placed as follows:
\startTEX
\IncludePicture
  [horizontal]
  [filename]  % Name of the file that contains the picture
  {Title of the slide}
\stopTEX

while a vertical picture is placed as follows:
\startTEX
\IncludePicture
  [vertical]
  [filename]  % Name of the file that contains the picture
  {Text that is placed on the right of the picture}
\stopTEX

It is possible to change the height and width of the pictures, or 
highlight them with circles and arrows. These details can be found in \in
{Section}[sec:pictures]

\page

\section[sec:styles]{Changing presentation styles}

The \options{style} key to \typeTEX{\setupmodule[simpleslides]} determines the
look of the presentation. Some of the styles come with variants, that can be
chosen using \options{color} and \options{alternative} keys. The available
styles are shown below along with the details of their variants.

\subsubject{BigNumber: with \options{color=blue} (also accepts \options{color=red})}

This is a style with subdued and quiet colors; its characteristic feature is the
page number on the lower right border of the text area. This detail was inspired
by the {\em split} style (\filename{s-pre-14}) by Hans.

\ShowStyle {BigNumber-blue}
\page

\subsubject{BottomSquares}

This minimalistic style is inspired by a presentation Taco gave at EuroTeX
2006.

\ShowStyle {BottomSquares}
\page

\subsubject{Boxed}

This style is inspired by the screen version of the Metafun manual. Watch
the small gray circles at the bottom!

\ShowStyle {Boxed}
\page

\subsubject{Ellipse}

This style is inspired by {\em funny} style (\filename{s-pre-03}) by Hans. 
The light red stripe marks the progress.

\ShowStyle {Ellipse}
\page

\subsubject{Embossed}

Spread the word, don't be shy! Show your pride in using \CONTEXT. The color
theme will probably look familiar; we copied it from the \filename{enattab}
manual. 

\ShowStyle {Embossed}

If you are shy, or narcissistic, you can change the emblem by
\startTEX
\setuplabeltext  [simpleslidesemblem={I made this presentation}]
\stopTEX

\page

\subsubject{Framed: with \options{alternative=square}}

This style was inspired by the {\em green} style (\filename{s-pre-02}) by
Hans. It has a thick blue frame around the entire slide area and a thinner
frame around the text area. The style has two options for alternative:
\options{alternative=stripe} will display a shaded blue area which will
grow with each slide; \options{alternative=square} displays a row of blue
squares at the bottom which also measure the presentation's progress.

\ShowStyle {Framed-square}
\page

\subsubject{Framed: with \options{alternative=stripe}}
\ShowStyle {Framed-stripe}
\page

\subsubject{FramedTitle}

This is a style with loud titles. Its characteristic feature is the {\em scratch
counter} at the bottom, which is derived from Section~7.2 of the Metafun
manual.

\ShowStyle {FramedTitle}
\page

\subsubject{HorizontalStripes: with \options{color=green} (also accepts
\options{color=blue} and \options{color=red})}

A sober style with an emphasis on horizontal lines, inspired by the {\em
Szeged} theme in \LATEX's \filename{beamer} package.

\ShowStyle {HorizontalStripes-green}
\page

\subsubject{NarrowStripes: with \options{color=green} (also accepts
\options{color=blue} and \options{color=red})}

A very simple and sober style, with shaded narrow stripes. 

\ShowStyle {NarrowStripes-green}
\page

\subsubject{RainbowStripe}

A colorful style for daring presenters. The black line which marks the
progress is reminiscent of absorption lines in star spectra, so this style
may be apt for astrophysical presentations?

\ShowStyle {RainbowStripe}
\page

\subsubject{Rounded}

This style has cool colors and lots of white space; it is probably best suited
for presentations with relatively little text.

\ShowStyle {Rounded}
\page

\subsubject{Shaded: with \options{color=blue} (also accepts
\options{color=green} and \options{color=bluered})}

The only ornament to this style is the dark shaded background. It uses
\CONTEXT's \type{interactionbar} mechanism to show the progress of the
presentation. It provides much space for text.

\ShowStyle {Shaded-blue}
\page

\subsubject{SideSquares}

This style is inspired by the colors and corporate look of Thomas's 
university. It is very sober and offers much space for text and
images. There is a rough progress meter built into the blue quadrangles. 

\ShowStyle {SideSquares}
\page

\subsubject{SideToc}

This Style has a list of Topics in its left margin; the current topic is
automatically highlighted. To set a topic and add it to this table simple
type \typeTEX{\Topic[TopicName]} in your source file where the new topic
begins. 

\ShowStyle {SideToc}
\page

\subsubject{Split}

This style is inspired by the {\em Copenhagen} theme of the \LATEX's
\filename{beamer} package. The narrow blue and black stripes at the top and the
bottom of the slides display the date and slide number (top) and the title
and author of the presentation. 

\ShowStyle {Split}
\page

\subsubject{Sunrise}

This style is inspired by the {\em husky} theme of the \LATEX's
\filename{powerdot} package.

\ShowStyle {Sunrise}
\page

\subsubject{Swoosh}

Take a break from the right angles and straight lines. Use swooshy curves. This
style also has a fancy page counter at the bottom.

\ShowStyle {Swoosh}
\page

\subsubject{ThickStripes}

This theme is inspired by the {\em Berkeley} style of the \LATEX's
\filename{beamer} package. It has a stop watch at the bottom, which keeps track
of the number of slides.

\ShowStyle {ThickStripes}
\page

\section[sec:fonts]{Changing presentation fonts}

The \options{font} and the \options{size} keys to
\typeTEX{\setupmodule[simpleslides]} determine the font and font size for the
main text of the presentation. The default font is Latin Modern Sans at 17pt. 

\startitemize
\item The \options{font} key can take the following values.

\starttabulate[|l|p|]
\NC \options{LatinModern}     \NC typesets in Latin Modern Serif     \NC \NR
\NC \options{LatinModernSans} \NC typesets in Latin Modern Sans      \NC \NR
\NC \options{Bookman}         \NC typesets in \TEX Gyre Bonum (a Bookman
    clone)       \NC \NR
\NC \options{Chancery}        \NC typesets in \TEX Gyre Chorus 
    \footnote{Please be aware that Chorus is a calligraphic font. It has no
    italic or bold.} (a Zapf Chancery clone)                        \NC \NR
\NC \options{Gothic}          \NC typesets in \TEX Gyre Adventor (a Gothic
    clone)   \NC \NR
\NC \options{Helvetica}       \NC typesets in \TEX Gyre Heros (a Helvetica
    clone)       \NC \NR
\NC \options{Palatino}        \NC typesets in \TEX Gyre Pagella (a Palatino
    clone)   \NC \NR
\NC \options{Schoolbook}      \NC typesets in \TEX Gyre Schola (a Schoolbook
    clone)     \NC \NR
\NC \options{Times}           \NC typesets in \TEX Gyre Termes (a Times clone)
\NC \NR
\stoptabulate

\item The \options{size} key can be any valid \TEX\ dimension.

\stopitemize

\subsubject{Choosing your own font}

If you want to set up your own font, pick any value for the \options{font} key
(or leave it empty). Use the \options{size} key to choose the font size. Then
{\em after} loading the module, choose any font using the normal \CONTEXT\
commands. Make sure to set the bodyfont at size \typeTEX{\NormalSize}. So, if
you have your own typescript for a font, your setup will look like this:

\startTEX
\usemodule[simpleslides][...]
....
\usetypescriptfile[type-myfont]    % The typescript for your font
\usetypescript[Mytypescript]       % As set in your typescript file
\setupbodyfont[Myfont,\NormalSize] % Note the \NormalSize here
\stopTEX

Internally, the font size is stored in the macro \typeTEX{\NormalSize}. The main
text is set at size \typeTEX{\NormalSize}; the main title is set at
\typeTEX{\TitleSize} while the author and date on the title page, and the slide
title are set at \typeTEX{\SlideTitleSize}. 

\typeTEX{\NormalSize}, \typeTEX{\TitleSize}, and \typeTEX{\SlideTitleSize} are
defined in terms of the dimensions \typeTEX{\simpleslidesNormalSize},
\typeTEX{\simpleslidesTitleSize}, and \typeTEX{\simpleslidesSlideTitleSize}.
\typeTEX{\simpleslidesNormalSize} is equal to the \options{size} option. The
module uses some heuristics to select a reasonable value of
\typeTEX{\simpleslidesTitleSize} and \typeTEX{\simpleslidesSlideTitleSize}. If
you do not like the size of the title page and slide titles, you can change
their value to whatever you like.


\section[sec:setuptitle]{Changing the title page}

It is possible to change the look of \typeTEX{\placeTitle} using
\typeTEX{\setupTitle}. This feature is intended for authors creating a new
style, but may also be useful for someone who likes to tweak the presentation
style. You should normally only set the  \options{title}, \options{authors}, and
\options{date} keys. If \options{date} is not set, then the module will default
to \typeTEX{\currentdate}.

\setup{setupTitle}

\section{Changing the slide titles}

It is possible to change the look of \typeTEX{\SlideTitle} using
\typeTEX{\setupSlideTitle}. Like \typeTEX{\setupTitle}, this feature is intended
for authors creating a new style. You can use this command to make a minor
change in an existing style, if you want.

\setup{setupSlideTitle}

\section[sec:pictures]{Special macro for including pictures}

As explained earlier, the \typeTEX{\IncludePicture} macro facilitates the
placement of pictures. It takes four arguments (one of which is optional, and as
such wasn't mentioned in the previous description).

\setup{IncludePicture}

As explained earlier, the first argument determines whether the picture will be
placed in horizontal or vertical layout; for examples, see \in
{Figure}[fig:pictures]. The second argument is the filename of the picture that
you want to include. The third argument is an optional argument useful for
highlighting the picture. The fourth argument (in braces) is the text
accompanying the picture. For horizontal pictures, this text is placed as a
\typeTEX{\SlideTitle}; for vertical pictures this text is placed opposite to the
picture, centered horizontally and vertically. 

The third argument is the most complex. It specifies picture dimensions and
highlights. If you want all pictures to share a common value (like
\options{color} or \options{shadow}), specify them using
\typeTEX{\setupPicture}.

\setup{setupPicture}

\null\blank
Below is a brief explanation of what the different parameters do:

\startitemize[packed]
  \item \options{width} and \options{height} \par
    Unsurprisingly, these set the width and height of the picture. Normally, the
    module will automatically scale your pictures to fill the available space, so
    you only need to set one of these values if you want to override this
    mechanism. 

  \item \options{highlight} \par
    This key determines the highlighting of the picture. If you set
    \options{highlight=yes}, then you can use one of the three available
    highlights: \options{circle}, \options{arrow}, and \options{focus}. These
    highlights are shown in \in{Figure}[fig:highlight]. The specific highlight is
    chosen using the \options{alternative} key. The location of the highlight is
    specified using the \options{x} and \options{y} keys. The scaling and
    rotation of the highlights is set using \options{xscale}, \options{yscale},
    \options{length} and \options{direction}.

  \item \options{alternative} \par
    When \options{highlight=yes}, three different highlights are
    available: \options{circle}, \options{arrow}, and \options{focus}.
    
\stopitemize

\placefigure
  [top,bottom]
  [fig:highlight]
  {Different highlight options available}
  \startcombination[2*2]
    {\externalfigure[styles/BigNumber-blue][page=3,width=0.55\textwidth]}
    {Picture with \options{highlight=no} (default)}
    {\externalfigure[styles/BigNumber-blue][page=7,width=0.55\textwidth]}
    {Picture with \options{highlight=yes} and \options{alternative=circle}}
    {\externalfigure[styles/BigNumber-blue][page=8,width=0.55\textwidth]}
    {Picture with \options{highlight=yes} and \options{alternative=arrow}}
    {\externalfigure[styles/BigNumber-blue][page=9,width=0.55\textwidth]}
    {Picture with \options{highlight=yes} and \options{alternative=focus}}
  \stopcombination

\subsubject{Units for dimensions}

All dimensions are specified relative to the width and height of the image, so
you do not have to change the location of your highlights if you change the
presentation style. The dimensions \options{x} and \options{y} should be a
number between 0 and 10. The \options{x} is scaled by 1/10 times the width of
the image; the \options{y} value is scaled by 1/10 times the height of the
image. The easiest way to understand this is to look at a scaled grid
superimposed on the picture, as in \in{Figure}[fig:grid]. The grid is configured
as follows:

\startitemize[packed]
  \item \options{grid} and \options{subgrid} \par
    These options determine whether or not to show the grid and sub-grid. The
    \options{grid} divides the height and width of the picture into 10 sections;
    this is helpful for determining the exact position where you want to place
    circles and arrows. The \options{subgrid} divides the grid into a finer
    grid. Each cell is divided into \options{steps} times \options{steps} cells.

  \item \options{gridcolor} \par
    This option determines the color in which the grid is drawn. It can be any
    \CONTEXT\ color identifier. The default value is green.

  \item \options{steps} \par
    The number of subdivisions for the \options{subgrid}. The default value is
    5.
\stopitemize

\placefigure
  [top,bottom]
  [fig:grid]
  {Grids for help in determining the location of highlight}
  \startcombination[2]
    {\externalfigure[styles/BigNumber-blue][page=4,width=0.55\textwidth]}
    {Picture with \options{highlight=yes} and \options{grid=yes}}
    {\externalfigure[styles/BigNumber-blue][page=5,width=0.55\textwidth]}
    {Picture with \options{highlight=yes}, \options{grid=yes} and
    \options{subgrid=yes}}
  \stopcombination

\subsubject{Highlighting by a circle}

Now lets see how different highlight alternatives are specified. Suppose we want
to place the picture of a cow and highlight its head. To help determine the
center of the circle, we can first superimpose a fine grid on the picture, and
read the value for the center. From \in{Figure}[fig:grid], \options{x=1.4} and
\options{y=8.2} seems like a good value. Next we need to decide on the radius of
the circle. The radius can either be specified in terms of the \quotation{x
units} (1/10th of the picture width) or \quotation{y units} (1/10th of the
picture height). Lets try a radius of 1.5 \quotation{x units}. This can be
specified as \options{xscale=1.5}. If we wanted something in terms of
\quotation{y units}, we could have used \options{yscale}. If both
\options{xscale} and \options{yscale} are specified, we will get an ellipse. 
Thus, to draw the circle highlight shown in \in{Figure}[fig:highlight], we wrote

\startTEX
\IncludePicture
  [horizontal]
  [cow] % Name of the image
  [highlight=yes,
   alternative=circle,
   x=1.4,
   y=8.2,
   xscale=1.5,
   shadow=bottomleft]
  {The head of a dutch cow}
\stopTEX

If \options{direction} key is specified, the circle (or the ellipse) will be
rotated by that amount (in degrees) in the counter clockwise direction. The
color in which the circle is drawn is specified using \options{color} key. The
thickness of the line is determined by \options{rulethickness} key. By default,
\options{color=orange} and \options{rulethickness} is 1/100th of the picture
width.

In summary, the different keys related to \options{alternative=circle} are:
\startitemize[packed]
  \item \options{highlight=yes} and \options{alternative=circle} \par
    These are needed to specify a circle highlight.
  \item \options{x} and \options{y} \par
    The center of the circle in terms of scaled units. Their values should be
    between 0 and 10.
  \item \options{xscale} and \options{yscale} \par
    The radius of the circle if only one option is specified. The major and
    minor radii of the ellipse if both options are specified. 
  \item \options{direction} \par
    The amount by which the circle is rotated. This only makes sense if we are
    actually drawing an ellipse.
  \item \options{rulethickness} \par
    The line width of the circle.
  \item \options{color} \par
    The color of the circle.
\stopitemize

\subsubject{Highlighting by an arrow}

Suppose we want to include a picture of a cow and point out its mouth using an
arrow. An arrow is specified by three things, the location of its tip, given by
\options{x} and \options{y} keys; the length of the arrow, given by
\options{length} key; and the direction of the tail, given by
\options{direction} key. Thus, to draw the arrow highlight shown in \in
{Figure}[fig:highlight], we wrote
\startTEX
\IncludePicture
  [horizontal]
  [cow] % Name of the image
  [highlight=yes,
   alternative=arrow,
   x=0.4,
   y=6.8,
   direction=-90, 
   length=3cm,
   shadow=bottomleft]
  {The mouth of a dutch cow}
\stopTEX

The different keys related to \options{alternative=arrow} are:
\startitemize[packed]
  \item \options{highlight=yes} and \options{alternative=arrow} \par
    These are needed to specify a arrow highlight.
  \item \options{x} and \options{y} \par
    The tip of the arrow in terms of scaled units. Their values should be
    between 0 and 10.
  \item \options{length} \par
    The length of the arrow. This is a dimension.
  \item \options{direction} \par
    The amount by which the arrow is rotated. 
  \item \options{rulethickness} \par
    The line width of the arrow. (Actually the line width of the arrow is twice
    the given value. This is so that both arrows and circles look good with the
    same value of rulethickness.)
  \item \options{color} \par
    The color of the arrow.
\stopitemize

\subsubject{Highlighting by focus}

Suppose we want place the picture of a cow, focus its head, and dull out rest of
the picture. The area to be focused is a circle (or an ellipse) and it can be
specified using \options{x} and \options{y} to indicate the center,
\options{xscale} and \options{yscale} to indicate the radius, and
\options{direction} to indicate the rotation. The keys \options{rulethickness}
and \options{color} do not have any effect. The area other than the focussed
area is washed out with a transparent color. The degree to which it is washed
out is determined by \options{opacity} (default value 0.5), and the color of the
unfocussed area is determined by \options{shadowcolor} (default value black).
Thus, to draw the focus highlight shown in \in{Figure}[fig:highlight], we wrote

\startTEX
\IncludePicture
  [horizontal]
  [cow] % Name of the image
  [highlight=yes,
   alternative=focus,
   x=1.4,
   y=8.2,
   xscale=1.5]
  {The head of a dutch cow}
\stopTEX

The different keys related to \options{alternative=focus} are:
\startitemize[packed]
  \item \options{highlight=yes} and \options{alternative=focus} \par
    These are needed to specify a focus highlight.
  \item \options{x} and \options{y} \par
    The center of the circle in terms of scaled units. Their values should be
    between 0 and 10.
  \item \options{xscale} and \options{yscale} \par
    The radius of the circle if only one options is specified. The major and
    minor radii of the ellipse if both options are specified. 
  \item \options{direction} \par
    The amount by which the circle is rotated. This only makes sense if we are
    actually drawing an ellipse.
  \item \options{opacity} \par
    The opacity of the unfocussed area. \options{opacity=0} is transparent,
    while \options{opacity=1} is completely opaque.
  \item \options{shadowcolor} \par
    The color of the unfocused area.
\stopitemize

\subsubject{Adding shadows}

When a circle or arrow highlight is used, adding a shadow to the highlight makes
them stand out more. The key related to shadows is:
\startitemize[packed]
  \item \options{shadow} \par
    This key determines whether shadows are placed or not. By default, shadows
    are disabled. If not set to \options{no}, this key determines where the
    shadow is placed: at \options{topleft}, \options{topright},
    \options{bottomleft}, or \options{bottomright}. Setting this key to
    \options{yes} puts the shadow at \options{bottomright}.
\stopitemize

\subsubject{Adding a specific page}

To select a specific page from a multi-page pdf file, you can use
\options{page=<number>} option.

\subsubject{Using your own style}

The module makes it easy to write your own style or to tweak one of the
provided styles beyond the configuration options provided by the
module. Simply copy the style which is closest in appearance to what you
want to obtain. Give it a filename \filename{s-myownstyle.tex},
\filename{myownstyle} being any name you like. Put this file into a
directory where \CONTEXT\ will find it, either the directory where you will
process your presentation or somewhere in your personal \filename{$TEXMF}
tree. Then, let the module know that you want to use your own style:

\startTEX
\usemodule[simpleslides]
          [style=myownstyle]
\stopTEX

The module will read your file and apply your settings.

\stoptext

% vim: foldmethod=marker foldmarker=<<<,>>>
